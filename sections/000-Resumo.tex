
\renewcommand{\abstractname}{Resumo}
\begin{abstract}
  \noindent A taxa de atividades de materiais catalíticos sofre redução por
  diversos mecanismos, sendo um deles o envenenamento de regiões catalíticas por
  substâncias químicas adsorvidas, de forma que reagentes passam a não alcançar
  mais tais regiões. Este trabalho trata de uma simulação cujo principal objetivo
  é replicar resultados já alcançados em trabalhos prévios que, além de resultados
  oriundos de simulação, também mostram resultados analíticos aproximados. O
  modelo simulado trata de um sistema composto por uma rede unidimensional, com
  sítios catalíticos e reagentes uniformemente distribuídos, onde tal rede é palco
  de reações unimoleculares. A dinâmica do modelo se dá através da difusão dos
  reagentes que possivelmente atingem um sítio catalítico, reagem instantaneamente
  e deixam o sistema na forma de produto, sendo que essa reação pode desativar o
  sítio catalítico com probabilidade \textit{p}. O sistema apresenta 2 fases
  distintas: uma em que um número positivo finito de reagentes sobrevive à
  dinâmica; outra em que há desativação de todos os sítios catalíticos da rede. A
  dinâmica do sistema é regida por reação bimolecular, onde sítios catalíticos
  desempenham papel da segunda espécie reativa, quando na região de fronteira
  entre tais fases. O comportamento do sistema próximo da criticalidade apresenta
  cruzamento, onde ele obedece lei de potência com expoente próximo de
  $-1/4$ até tempos da ordem de $10^3$.
\hfill \break
\hfill \break
\textbf{Palavras-chave:} engenharia de software, redes discretas, análise estatística de simulações.
\end{abstract}

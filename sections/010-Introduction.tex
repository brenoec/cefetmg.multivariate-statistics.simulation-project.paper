
\section{Introdução}

\lettrine{A}{} engenharia de software é uma área do conhecimento relativamente
recente. O termo já era usado na década de 1950, apesar de que a primeira
conferência sobre o tema tenha ocorrido em Garmish, Alemanha, em 1968. A
jovialidade da área de conhecimento se deve ao fato de que o surgimento do
software se deu após se verificar, na década de 1940, a necessidade de se
flexibilizar a programação dos dispositivos computacionais\cite{Wazlawick2013}.

Os anos entre 1960 e 1980 foram marcados pela ``crise do software'', onde parte
considerável dos problemas relacionados à produção de software foram
indentificados. A produção de software foi enquadrada como uma engenharia e seus
processos foram sistematizados ao longo dos anos que se seguiram
\cite{Wazlawick2013}.

A prática da engenharia de software é considerada uma das tarefas mais complexas
exercidas por seres humanos\cite{Crockford2008}. Determinados tipos de sistemas
apresentam e exigem maior ou menor grau de complexidade e de confiabilidade, uma
vez que falhas de sistemas de software podem impactar em perdas de diferentes
tipos de capital.

A complexidade da produção do software não é limitada por sua aplicação em si.
Apresenta aspectos de ordem tecnológica, técnica, metodológica, política,
social, econômica, cultural e mais. Ademais, existe esfoço considerável para
manter atualizado uma série de informações para que artefatos de um sistema
possam coexistir de forma coerente. A produção de software é, nesse sentido,
um trabalho que atua em uma base de concretos abstratos, sendo indispensável
para a sua realização o entendimento amplo dos elementos relacionados. Trata-se
de operações sobre um espectro de informações; portanto.

Em consonância com tais informações, as macrotarefas da produção de sistemas de
software --- desenvolvimento, manutenção e evolução de software --- exigem
compreensão ampla para que o resultado esteja em conformidade com os fatores
externos e internos de qualidade definidos pela norma ISO 9126\cite{ISOIEC9126}.

Grande parte das vezes que a humanidade se depara com desafios e obstáculos,
ela aplica conhecimento, técnica e tecnologia para potencializar seu campo de
atuação. Nesse sentido, é razoável que façamos o mesmo para ampliar a forma como
interagimos com sistemas de software e o volume de informação que os circundam.
